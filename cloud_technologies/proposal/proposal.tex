\documentclass{article}
\setcounter{secnumdepth}{0}

%-----------------------------------------------PACKAGES-------------------------------------------------------------%
\usepackage{graphicx} %images
\DeclareGraphicsExtensions{.pdf,.png,.jpeg} % configures latex to look for the following image extensions
\usepackage{setspace} % allows for configuring the linespacing in the document
\usepackage{caption}
\usepackage{appendix}
\usepackage[a4paper, total={6in, 8in}]{geometry}
\usepackage[explicit]{titlesec}
\usepackage{hyperref}
\hypersetup{
    colorlinks,
    citecolor=black,
    filecolor=black,
    linkcolor=black,
    urlcolor=black
}
 

%-----------------------------------------------#########--------------------------------------------------------------%


%PREAMBLE
\author{Stephen Coady}
\title{The Automation of Infrastructure Orchestration and Application Deployment using Ansible and Docker Swarm}
\onehalfspacing

%CONTENT
\begin{document}
\maketitle
\thispagestyle{empty}
\newpage

\paragraph{Title:} 
\label{par:Title}
The Automation of Infrastructure Orchestration and Application Deployment using Ansible and Docker Swarm

\paragraph{Introduction:} 
\label{par:Introduction}
Application deployment is becoming ever-increasingly complicated. This research paper aims to look at modern tools such as Ansible and Docker Swarm to see how they can aid in deployment. 

\paragraph{Objectives:}
\label{par:Objectives}
The objective of this research paper is to show how a modern automation tool such as Ansible can be used to provision and orchestrate the application cluster. \newline
\newline
Some tasks it will be used to carry out are: 

\begin{itemize}
	\item Deployment of application servers
	\item Provisioning of the above servers
	\item Deployment of the application to the servers
\end{itemize}

Another aim of this research paper is to show how Docker, and more specifically Docker Swarm, can be used to deploy an application to a cluster of servers.
\newline
\newline
Some tasks it will be used to carry out are:

\begin{itemize}
	\item Creating a single point of contact with multiple application servers for easy management
	\item Creating an application that runs inside a container
	\item Creating a fault-tolerant application
	\item Easy scaling
\end{itemize}

The overall architecture of this system can be seen below in Figure \ref{fig:swarm}.

\begin{figure}[!hb]
\centering
\includegraphics*[width=\textwidth]{swarm}
\caption{\em A Docker Swarm with a Loadbalancer handling requests}
\label{fig:swarm}
\end{figure}

\paragraph{Outcome:}
\label{par:Outcome}
The outcome of this research paper will be an application which is fully distributed across multiple servers, is fault-tolerant and can be scaled reliably which has been completely deployed using the automation tool Ansible.



\end{document}
