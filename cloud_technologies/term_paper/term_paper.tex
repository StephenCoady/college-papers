\documentclass{article}

%-----------------------------------------------PACKAGES-------------------------------------------------------------%
\usepackage{graphicx} %images
\DeclareGraphicsExtensions{.pdf,.png,.jpg} % configures latex to look for the following image extensions
\usepackage{setspace} % allows for configuring the linespacing in the document
\usepackage{caption}
\usepackage{natbib}
\usepackage{appendix}
\usepackage[a4paper, total={6in, 8in}]{geometry}
\usepackage[explicit]{titlesec}
\usepackage{hyperref}
\usepackage[nottoc,numbib]{tocbibind}
\hypersetup{
    colorlinks,
    citecolor=black,
    filecolor=black,
    linkcolor=black,
    urlcolor=black
}
\bibliographystyle{agsm}
\setcitestyle{authoryear,open={(},close={)}}

%-----------------------------------------------#########--------------------------------------------------------------%


%PREAMBLE
\author{Stephen Coady}
\onehalfspacing

%CONTENT
\begin{document}
\begin{titlepage}
	{\huge\bfseries Lifecycle Management For Docker UI\par}
	\vspace{1cm}
	{\scshape\large Proposal \par}
	\vspace{6cm}
	{\Large Stephen Coady\par}
	{\Large Applied Computing\par}
	{\Large 20064122\par}
%	\vfill
%	supervised by\par
%	Dr.~Mark \textsc{Brown}
\vspace{2cm}

{\large \today\par}

	\vfill
	
% Bottom of the page
\end{titlepage}

\thispagestyle{empty}

\newpage
\tableofcontents
\newpage

\newpage
\section{Introduction} %REVIEW Not overly happy with this, needs refining and improving.
\label{sec:Introduction}
Modern applications are becoming increasingly complex, meaning it can also be complex to deploy the application. This research paper will examine application deployment, and will aim to show how modern tools and technologies can be used to simplify the process of building and deploying an application to the cloud. 

It will compare these tools with ``legacy'' processes, evaluating the strengths and weaknesses of both. It will do this under the premise of a problem domain, discussed in Section \ref{sec:Problem}.

Not only is application deployment a problem, but provisioning the servers which the application is hosted on is also something which needs to be considered. This paper will look at the automation of this process.

\newpage
\section{Problem Domain}
\label{sec:Problem}


\newpage
\section{Technology Background}
\label{sec:Background}

\subsection{Containers}
\label{sub:Containers}

\subsection{Docker}
\label{sub:Docker}

\subsubsection{Swarm}
\label{subs:Swarm}

\subsection{Ansible}
\label{sub:Ansible}


\newpage
\section{Building an Application}
\label{sec:Build}

\newpage
\section{Conclusion}
\label{sec:Conclusion}

\newpage
\bibliography{references}


\end{document}
