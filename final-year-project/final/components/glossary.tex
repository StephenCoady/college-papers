% !TEX root = ../final_report.tex
\newglossaryentry{sprint}
{
    name={Sprint},
    text={sprint},
    plural={sprints},
    description={A short cycle of development - normally one or two weeks long. Allows for rapid change of development direction if project requirements change}
}
\newglossaryentry{HTML}
{
    name={HTML},
    text={HTML},
    description={Hypertext Markup Language. A means of tagging text using ``markup'' to allow them to be positioned, styled and linked on web pages}
}
\newglossaryentry{JWT}
{
    name={JWT},
    text={JWT},
    description={JSON Web Tokens. A means to provide authentication over the internet using a simple string of characters}
}
\newglossaryentry{CSS}
{
    name={CSS},
    text={CSS},
    description={Cascading Style Sheet. Used for applying styles to markup text primarily displayed on web pages}
}
\newglossaryentry{code coverage}
{
    name={Code Coverage},
    text={code coverage},
    description={Refers to the percentage of code which has been executed by tests. Not an indication of test quality but instead can be used to identify pieces of code which has been missed by test cases}
}
\newglossaryentry{VM}
{
    name={Virtual Machine},
    text={virtual machine},
    description={An abstracted virtual version of a computer. Normally runs on another host with a dedicated piece of software to manage it. Appears to the user as a real computer, however all or most components are defined by software}
}
\newglossaryentry{Bootstrap}
{
    name={Bootstrap},
    text={Bootstrap},
    description={A front end css framework designed to give developers more power with less work when creating html pages. Makes use of a standard 12-width grid system to provide an easy-to-use css styling method}
}
\newglossaryentry{API}
{
    name={API},
    text={API},
    description={Application programming interface. A set of endpoints which lead to functions and application logic. Allows developers to expose application functionality without directly exposing the code within the application}
}
\newglossaryentry{Vagrantfile}
{
    name={Vagrantfile},
    text={Vagrantfile},
    description={A template file which provides Vagrant with all the instructions required to create the desired environment. Written in the Ruby programming language}
}
\newglossaryentry{Docker container}
{
    name={Docker container},
    text={Docker container},
    description={An isolated package which bundles everything required to run an application within a container. The application can then be run within this container without regard to the underlying architecture}
}
\newglossaryentry{Dockerfile}
{
    name={Dockerfile},
    description={A template from which a Docker image can be built. Defined in plaintext and interpreted by the Docker Engine. Can define which commands should be run etc}
}
\newglossaryentry{REST}
{
    name={REST},
    text={REST},
    description={Representational state transfer. A means of providing communication between computers across the internet. RESTful web services allow systems interact with each other using completely stateless operations}
}
\newglossaryentry{CLI}
{
    name={CLI},
    text={CLI},
    description={Command Line Interface. A means of interacting with a computer program where a user can only enter commands in the form of successive lines of text}
}
\newglossaryentry{UI}
{
    name={UI},
    text={UI},
    description={User Interface. A graphical view towards an application which exposes functionality by using buttons or other components. Typically utilised using a computer mouse}
}
\newglossaryentry{Docker}
{
    name={Docker},
    text={Docker},
    description={An application which manages containers running on a host}
}
\newglossaryentry{CRUD}
{
    name={CRUD},
    text={CRUD},
    description={Create, Read, Update, Delete. CRUD defines the four cornerstones of API functionality. An API should provide these functions to any user using the API}
}
\newglossaryentry{container}
{
    name={Container},
    text={container},
    description={An isolated piece of software bundled with everything it needs to run on the host}
}
\newglossaryentry{docker}
{
		name={Docker},
    text={Docker},
		description={A piece of software which enables orchestration and management of containers on a host}
}
\newglossaryentry{Docker host}
{
		name={Docker host},
    text={Docker host},
		description={The computer, server or virtual machine on which the Docker application is installed. One user may manage several Docker host's, depending upon how many computers they have installed it on}
}
\newglossaryentry{Docker image}
{
		name={Docker image},
		description={A template from which many containers can be started. Can be pushed an pulled to/from a remote Docker registry}
}
\newglossaryentry{staging server}
{
		name={Staging Server},
		text={staging server},
		description={A computing unit exposed on the network which typically runs an incomplete or demo version of a piece of software. Generally used to show the in progress development to any product stakeholders}
}
\newglossaryentry{Docker daemon}
{
		name={Docker daemon},
		description={The server-side component of the Docker Engine}
}
\newglossaryentry{lifecycle management}
{
		name={Lifecycle Management},
    text={lifecycle management},
		description={The management of a container over its complete lifecycle, from its beginnings as an image to starting, stopping and restarting the container and managing the interactions it has with other containers on the system}
}
\newglossaryentry{ci}
{
		name={Continuous Integration},
    text={continuous integration},
		description={The act of continuously merging newly created software with the current working version, validated by automatic build and testing tools allowing for early detection of faults}
}
\newglossaryentry{continuous deployment}
{
		name={Continuous Deployment},
		text={continuous deployment},
		description={The act of continuously deploying software which has passed acceptance tests to a live system}
}
\newglossaryentry{open source}
{
    name={Open Source},
    text={open source},
    description={Software which encourages input and feedback from other developers and allows for any and all modifications to the current software package with the intent to release it under a new package}
}
\newglossaryentry{technical debt}
{
    name={Technical Debt},
    text={technical debt},
    description={The extra development work which arises as a result of implementing an easier but less robust solution}
}
\newglossaryentry{sonarqube}
{
    name={SonarQube},
    text={SonarQube},
    description={A tool to scan source code. It can inform developers of coding best-practices and highlight ``code smells'', which are essentially when best practices are not adhered to. Also highlights technical debt. Code is deemed either `passing' or `failing', meaning the code has been deemed either sufficient or insufficient in the following categories: Code Test Coverage, Bugs, Vulnerabilities and Technical Debt Ratio}
}
\newglossaryentry{code smell}
{
    name={Code smell},
    text={code smell},
    description={A piece of code which may perform as expected in some or all cases but that introduce a possibility of an unexpected result. Should be avoided. They are normally decided upon by the community}
}
\newglossaryentry{Travis}
{
    name={Travis},
    text={Travis},
    description={A website which allows for automatic builds of code and feedback of results. Can be used to automatically deploy code dependent on test results also}
}
\newglossaryentry{git}
{
    name={Git},
    description={A version control system which allows source code to be tightly controlled and stored in a central repository}
}
\newglossaryentry{github}
{
    name={Github},
    text={Github},
    description={An online platform to store git repositories. Allows for multi-developer collaboration on projects}
}
\newglossaryentry{dockerhub}
{
    name={DockerHub},
    text={DockerHub},
    description={An online platform to store Docker images. Can be pulled from by anyone once the repository is public. Essentially a way to share any built Docker images}
}
\newglossaryentry{Docker Compose}
{
    name={Docker Compose},
    description={A tool developed by Docker to allow developers to easily orchestrate multi-container systems without requiring them to manually manage containers}
}
