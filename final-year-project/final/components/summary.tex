% !TEX root = ../final_report.tex
\section{Summary}
\label{sec:summary}
In this Section the aim is to dissect the project, examine its proposed deliverables and compare them to the components it actually delivered. This will determine the overall success of the project. The direction of the project going forward will also be discussed in terms of the product backlog and the work which is still outstanding.

\subsection{Review}
At the beginning of the project all stakeholders decided on the core and stretch goals of the system - these would then form the initial requirements. This list was then broken down further and was the basis for the initial product backlog. This high-level list can be seen below in Figure \ref{fig:requirements}.

\begin{figure}[!ht]
\centering
\includegraphics*[width=0.6\textwidth]{images/requirements}
\caption{\em Initial Project Requirements}
\label{fig:requirements}
\end{figure}

It is clear that all core requirements were delivered and only one stretch goal was not delivered. However as this is now an open source project the tasks associated with this feature have been created and added to the product \gls{backlog} which can be seen in Section \ref{subs:cont_dev}.

The main three components this project has delivered will now be focused on. As the system is loosely coupled each component can be completely recycled in any other project or indeed be replaced within this project by a comparable component.

\paragraph{Server Side Application}\mbox{}\\
The server side of this application is a node module which communicates with the Docker API by using the third party module, Dockerode, previously mentioned in Section \ref{sec:design}.

\paragraph{Independent API}\mbox{}\\
The Express JS API has been designed to be independent of any server side or front end code. While it is run by the node module it is not constrained by it. An API written in any other language could easily be placed in front of the node module also. 

\paragraph{Front End Application}\mbox{}\\
The front end application written using Angular is probably the most weakly coupled component delivered. It is a web application which consumes API endpoints and could just as easily be written using another front end framework.

\subsection{Learning Outcomes}
The learning outcomes of this project can be categorised under technical and non-technical.

\subsubsection{Technical}
As a result of developing this application many technical skills were gained. These include:

\begin{itemize}
	\item Gained a deep understanding of container technologies, both on the command line and by using the APIs exposed by services such as Docker.
	\item Learned how to use open source software effectively. i.e while using a package it may be necessary to git clone the software and run it locally. Once software has been run locally it makes it much easier to interact with the community by asking questions.
	\item Developing an application which has some elements of all modern applications - i.e a server side component, a database, an API and a front end - has resulted in valuable experience of developing the `full-stack'.
	\item Using current best practices by creating a fully-functional continuous integration and deployment pipeline has enforced strict coding practices which is invaluable when working as part of a multi-developer team.
	\item Using code quality tools such as SonarQube has meant code quality is no longer an afterthought when developing applications. This ultimately means a better and more reliable software developer. 
\end{itemize}

\subsubsection{Non-Technical}
\begin{itemize}
	\item Working as part of an agile team, even as a lone developer, has meant increased awareness of the overall goal within a project and therefore a more focused method of development.
	\item Working with a \gls{Scrum} master has also enforced the ideology of performance and code reviews, again contributing to a better overall development experience and therefore a higher quality end product.
	\item Receiving input and advice throughout the project from product owners Red Hat mobile has been instrumental to professional development. It has lead to a more focused and organised process. Therefore the quality of the end product has also increased exponentially as a result.
	\item Communication skills have been greatly improved as a result of constant contact with both the product owner and \gls{Scrum} master.
\end{itemize}

\newpage
\subsection{Project Direction}
As this product was developed from the very beginning under the open source software ethos it is currently still being developed. The current direction can be categorised under the following headings.

\subsubsection{Continued Development}
\label{subs:cont_dev}
Currently the product backlog can be found \href{https://issues.jboss.org/secure/RapidBoard.jspa?rapidView=3836&view=planning}{here} and currently is as follows:

\begin{figure}[!ht]
	\centering
	\includegraphics*[width=0.9\textwidth]{images/final_backlog}
	\caption{\em Current Backlog}
	\label{fig:final_backlog}
\end{figure}

At the current project velocity this backlog contains approximately two sprints. It will implement several features which will improve the overall usability of the application. 

By creating a cumulative flow diagram it is clear how much progress has been achieved to date and also how much is remaining. This gives a good indication of time remaining before all items on the backlog are complete if current velocity is maintained. This is evident in Figure \ref{fig:cumulative_flow}.

\clearpage

\begin{figure}[!ht]
	\centering
	\includegraphics*[width=\textwidth]{images/cumulative_flow}
	\caption{\em Cumulative Flow Diagram}
	\label{fig:cumulative_flow}
\end{figure}

This graph shows that as an Agile project evolves, more work naturally is identified and scoped out which allows the backlog to grow. It facilitates natural ideation and innovation within a team once the work generated aligns to the project vision or goal. As such, a critical analysis of the priority of work has to occur to ensure that the work performed is valuable and timely. With the right work completed at the right time, it inevitably leads to stagnation of some backlog items. Those items are still valuable and it gives the developer a chance to define a critical path based on value which in turn allows for milestone releases to occur. So while the project is complete from the point of view of this final year project, the lifecycle of the overall project is still not complete and may never be complete. This is by design. As an open source project, with more enhancements and future use cases, the backlog will inevitably grow. This chart shown in Figure \ref{fig:cumulative_flow}, coupled with other backlog metrics, will thus help guide future developers on the next critical path and proved to be a good milestone indicator as this project progressed.

\subsubsection{Encourage Developer Contributions}
As Docker grows in popularity so too will the interest in peripheral tools and so this project may gain interest due to this. However, to raise its profile further steps like posting on developer community message boards or using social media to attract developers may be viable.

\subsubsection{Increase the Feature Set}
While the current feature set is large it could be increased further to provide users with more power from the UI. The Docker remote API is extremely large, therefore the amount of possible actions is only limited by the functionality provided by this remote API.

\subsection{Reflection}
From a personal perspective this project has been a huge milestone. It has shown me how the building blocks I have been learning about for the past four years can be put together to build one large, cohesive project. While it is a technical process it is also ultimately about the journey of creating this application. Starting with nothing and self-managing to the point where the end result is a fully functional application has taught me a lot about project and time management. This is an extremely valuable skill and one which I could not have honed without this project as a platform.

Developing the early prototype model in semester one also taught me about investing an initial amount of time into planning and feasibility investigation. In this project this was invaluable as it highlighted some strengths of the technology choice and also unearthed some which could have been costly if not caught so soon. To the untrained developer an initial technical feasibility may seem like a waste of time but this project has certainly shown me otherwise. 

While developing this project the act of constant self-evaluation paired with always assessing the priority of tasks has shown me how to be the most productive developer possible. This is an incredibly useful skill as it is vital in industry to retrospectively evaluate oneself. Without this skill it is difficult to learn from one's own mistakes, no matter how small or seemingly trivial those mistakes were.

The experience of having technical experts (in the shape of Red Hat) to consult with and seek advice from was also invaluable. I would encourage any future student who is given this opportunity to take it with both hands. I feel I made the most of this relationship by being proactive in my interactions with Red Hat. My experience has shown that companies are not afraid to engage with students - ultimately on the students own terms - and this can, in the end, be beneficial to both the student and the company involved.

I found that implementing to a strict timeline in the form of sprints was a major factor in the success of this project. By initially setting the amount of time (2 weeks) and exactly how many sprints there would be I created a finite timeline which needed to be worked through. The danger of a project like this in a college setting is that the student may devote too much or too little to the project dependent on other subjects. If this happens either the project or the other subjects are bound to suffer and by creating a definite timeline of 10 weeks in which all development will take place this risk was minimised for me. At the beginning of this project Dr. Brenda Mullally and I travelled to the Red Hat offices where all three parties worked through and agreed on the initial high level tasks of this project. This initial backlog grooming also included high level college-related tasks such as this report. In my opinion this, paired with the previously mentioned definite timeline, was a contributing factor to the overall quality of this project. By explicitly stating all tasks, including all those which are documentation-related meant a well structured project which in turn lead to a better project.

Overall I personally would deem this project a complete success. Without a doubt I have grown as a developer during the past 8 months as a result of the work I have completed but ultimately this project was the final learning outcome of the past four years.
