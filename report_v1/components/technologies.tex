\section{Technologies}
\subsection{Node.JS}
Node.js is a server-side JavaScript runtime, it is built on the same V8 engine that powers the popular Chrome browser. It uses an event-driven, non-blocking I/O model that makes it lightweight, efficient and very fast. Node.js' package ecosystem, npm, is the largest ecosystem of open source libraries in the world. \citep{Nodejs.org2016}.

It is node's asynchronous nature which makes it so powerful. Being asynchronous essentially means that it does not create a new thread every time a request needs to be dealt with. Instead, it relies on a single event loop to handle requests which never blocks I/O. This can be seen in Figure \ref{fig:event_loop}.

One benefit of Node using a single event loop is that it is less CPU intensive since it does not need a new thread for every new request. Node is also non-blocking, which when combined with using a single event-loop mean it is very fast.

Some advantages of using Node for this project are:

\begin{itemize}
  \item Large online community - vital for troubleshooting
  \item One of the largest online ecosystems providing a vast amount of third party modules, meaning it will not be necessary to ``re-invent the wheel''
\end{itemize}

\begin{figure}[!h]
\centering
\includegraphics*[width=0.6\textwidth]{images/event_loop}
\caption{\em Node Event Loop}
\label{fig:event_loop}
\end{figure}

\subsection{Mongo}
MongoDB is an open source document-oriented database. It is a NoSQL database which means Mongo does not support relationships between tables \citep{Mongo216}. Instead, MongoDB uses JSON-like ``documents'' to store information in an ordered way. This is advantageous for a number of reasons, namely:

\begin{itemize}
  \item Flexibility - Since Mongo does not have relations it is easy to change the database as development continues, without the need for migrations of the current schema.
  \item Javascript - Since the application itself will be written in Node it will be extremely beneficial to have a database which stores its information in JSON.
  \item Simplicity - When a database is schema-less it removes the need for complex actions such as joins, while still providing the power of complex queries.
\end{itemize}

\subsection{Supporting Technologies/Processes}
\label{subs:support}
\paragraph{Vagrant}\mbox{}\\
Vagrant is a development tool to provision and `sandbox' the complete development environment from the host machine it is running on. This is very beneficial, as it means all necessary dependencies can be bundled into one virtual machine and updated or changed when needed. 

Vagrant also allows the developer to standardize all of the different environments the application will run on. This means that if the code will be deployed to a specific version of a specific operating system then the developer can easily replicate this on the local development environment. It allows the developer to build the application on one (local) environment and deploy to the \textit{exact same} remote environment. This is extremely beneficial in terms of code reliability and stability - as the runtime can be reproduced easily.
\paragraph{Ansible}\mbox{}\\
Ansible is a tool to provision virtual machines running locally or remotely. A formal process of all deployment and provisioning will be employed in this project. Meaning that the same route from running code locally to it running on a remote server will always be followed.

Ansible paired with Vagrant in this project will mean any discrepancies or bugs in the code will not be introduced by the environment or the deployment/provisioning process. This is very important as it can reduce the development workload. 

\paragraph{Testing}\mbox{}\\
This project will use third party node modules to perform its unit tests. These unit tests will be behavioral-driven, meaning the components will be tested individually with the end-goal of the passing test being validating a certain behavior of the application, as opposed to validating the functionality of each individual component.

In this way, the project will follow the behavioral driven development (BDD) model. The advantage of this model over pure unit testing in test-driven development (TDD) is that BDD keeps the end result in mind at all times, instead of just the conditions for an individual unit test to pass. This allows the developer to constantly keep in mind the goal of the module being tested.
